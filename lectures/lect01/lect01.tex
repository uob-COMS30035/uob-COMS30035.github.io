\documentclass[10pt]{beamer}

\newcommand{\lectnum}{L01}
\newcommand{\lecttitle}{Unit organisation}

\usepackage{amsmath, amssymb, graphicx}
\usepackage[]{algorithm2e}
\usepackage{pdfpages}
\usepackage[british]{babel}

\hypersetup{colorlinks,linkcolor=,urlcolor=blue}
\newenvironment{titledslide}[1]{\begin{frame}\frametitle{#1}}{\end{frame}}

\mode<presentation>{\setbeamercovered{transparent}}

\setbeamertemplate{sidebar right}{}
\setbeamertemplate{footline}{%
\hfill\usebeamertemplate***{navigation symbols}
\hspace{0.4cm}\lectnum: \insertframenumber{}/\inserttotalframenumber \hspace*{0.4cm}}

\author{James Cussens}

\title{COMS30035, Machine learning:\\ \vspace{5pt} \lecttitle}

\institute{School of Computer Science\\University of Bristol}

\begin{document}
%%%%%%%%%%%%%%%%%%%%%%%%%%%%%%%%%%%%%%%%%%%%%%%%%%%%%%%%%%%%%%%%%%%%%%

\begin{frame}
  \titlepage
\end{frame}

%%%%%%%%%%%%%%%%%%%%%%%%%%%%%%%%%%%%%%%%%%%%%%%%%%%%%%%%%%%%%%%%%%%%%%


%%%%%%%%%%%%%%%%%%%%%%%%%%%%%%%%%%%%%%%%%%%%%%%%%%%%%%%%%%%%%%%%%%%%%%
\begin{titledslide}{Unit github page}

  \begin{itemize}
  \item Most teaching materials and information are accessed via the
    unit's github page:
    \href{https://uob-coms30035.github.io/}{https://uob-coms30035.github.io/}
  \item This also gives you the unit schedule.
  \item There is a Blackboard page, but this is used only for quizzes.
  \end{itemize}
  
\end{titledslide}
%%%%%%%%%%%%%%%%%%%%%%%%%%%%%%%%%%%%%%%%%%%%%%%%%%%%%%%%%%%%%%%%%%%%%%
\begin{titledslide}{Teaching staff}

  \begin{itemize}
  \item \href{https://jcussens.github.io/}{James Cussens} is the only
    lecturer for this unit.
  \item The following people are GTAs on the unit: Akin Eker, Siddhant
    Bansal, Shijia Feng, Omar Emara, Saptarshi Sinha, Tianye Wang and
    Kal Roberts.
  \item I give all the lectures.
  \item We are all present at drop-in sessions and labs.
  \end{itemize}
  
\end{titledslide}
%%%%%%%%%%%%%%%%%%%%%%%%%%%%%%%%%%%%%%%%%%%%%%%%%%%%%%%%%%%%%%%%%%%%%%
\begin{titledslide}{Teams}

  \begin{itemize}
  \item There is a Teams group for this unit. It has 2 public
    channels:
    \begin{description}
    \item[General] Read-only for students. (We might use this for
      announcements, but we can also use Blackboard for that.)
    \item[Open] Students can post here. Feel free to use it as you
      wish; for example to ask questions.
    \end{description}
  \item There is a separate Teams channel for students doing this unit
    as a MAJOR which is set up similarly and will be used for
    MAJOR-specific stuff (such as the coursework).
  \end{itemize}
  
\end{titledslide}
%%%%%%%%%%%%%%%%%%%%%%%%%%%%%%%%%%%%%%%%%%%%%%%%%%%%%%%%%%%%%%%%%%%%%% 
\begin{titledslide}{Teaching sessions}

  \begin{itemize}
  \item Lectures on Mondays and Wednesdays (both at 0900).
  \item Drop-in session on Thursdays (1200) followed, from Week 3
    onwards, by a 3-hour lab starting at 1500.
  \item During week 6 there is no teaching and during the coursework
    period (weeks 9-11) there is no teaching apart from coursework
    support sessions (1500-1700 Thursdays).
  \end{itemize}
  
\end{titledslide}
%%%%%%%%%%%%%%%%%%%%%%%%%%%%%%%%%%%%%%%%%%%%%%%%%%%%%%%%%%%%%%%%%%%%%%
\begin{titledslide}{Special arrangements for Weeks 1 \& 2}

  \begin{itemize}
  \item Since the Linux lab in QB1.80 did not become ready in time,
    the labs for the first two weeks will be in \textbf{MVB1.15}.
  \item In addition, these labs will only last 2 hours and will start
    at \textbf{1600} not 1500.
  \item MVB1.15 has only 75 computers and there are 105 students on
    this unit, so some of you may have to share with a friend if
    everyone turns up and uses the MVB1.15 computers (which is unlikely).
  \end{itemize}
  
\end{titledslide}
%%%%%%%%%%%%%%%%%%%%%%%%%%%%%%%%%%%%%%%%%%%%%%%%%%%%%%%%%%%%%%%%%%%%%%
\begin{titledslide}{Quizzes and problems}

  \begin{itemize}
  \item There are Blackboard quizzes and exercises associated with the
    lecture material.
  \item Pointers to these are included at the end of the slides for
    the relevant lecture.
  \item These are \textbf{formative} not \textbf{summative}, i.e.\
    they do not count towards your mark for this unit.
  \end{itemize}
  
\end{titledslide}
%%%%%%%%%%%%%%%%%%%%%%%%%%%%%%%%%%%%%%%%%%%%%%%%%%%%%%%%%%%%%%%%%%%%%%
\begin{titledslide}{Drop-in sessions}

  \begin{itemize}
  \item You are not obliged to attend any drop-in sessions.
  \item You can get help with the quizzes and exercises by coming to
    a drop-in session.
  \item But you can also ask us about anything at the drop-in sessions,
    e.g.\ further explanation of lecture material, pointers to
    additional reading, etc. 
  \end{itemize}
  
\end{titledslide}
%%%%%%%%%%%%%%%%%%%%%%%%%%%%%%%%%%%%%%%%%%%%%%%%%%%%%%%%%%%%%%%%%%%%%%
\begin{titledslide}{Assessment}

  \begin{itemize}
  \item If you are doing this unit as a MAJOR (COMS30083), then you
    are assessed by a mid-term in-class test in week 6 (30\%) and
    coursework (70\%) done during weeks 9-11.
  \item If you are doing this unit as a MINOR (COMS30081), then you
    are assessed by answering Machine Learning questions (equivalent
    to a 1-hour exam) in the \emph{Topics in Computer Science}
    December exam. This is a closed-book exam.
  \end{itemize}
  
\end{titledslide}
%%%%%%%%%%%%%%%%%%%%%%%%%%%%%%%%%%%%%%%%%%%%%%%%%%%%%%%%%%%%%%%%%%%%%%
\begin{titledslide}{Labs}

  \begin{itemize}
  \item During lab sessions you do machine learning exercises.
  \item You can use either the machines in the room or your own
    machine.
  \item The machines in the room have the necessary ML software
    installed.
  \item You need to take certain steps to access that software: see
    the section on Lab Work on the
    \href{https://uob-coms30035.github.io/}{unit github page}.
  \item If you choose to use your own machine you will have to install
    the necessary software yourself. This is entirely doable and we're
    happy to attempt to help you if you hit installation problems, but
    we can't guarantee to be able to fix all such problems
    (particularly if you choose to use Windows).
  \item Labs are the most important part of the unit since there you
    actually \textbf{do} machine learning, rather than merely hear
    about it.
  \end{itemize}
  
\end{titledslide}
%%%%%%%%%%%%%%%%%%%%%%%%%%%%%%%%%%%%%%%%%%%%%%%%%%%%%%%%%%%%%%%%%%%%%%
\begin{titledslide}{Python}

  \begin{itemize}
  \item All your coding on this unit will be done in Python.
  \item This code will call ML library methods supplied by packages
    such as scikit-learn, pytorch and PyMC.
  \item (Note that Python is mostly used a wrapper; the native code
    called by Python was typically compiled from C/C++/FORTRAN source.) 
  \item Your code will either be a regular script or in a JupyterLab
    notebook.
  \item Python coding, per se, is not taught on this unit. 
  \end{itemize}
  
\end{titledslide}

\end{document}
