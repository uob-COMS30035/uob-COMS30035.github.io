\documentclass[10pt]{beamer}

\newcommand{\lectnum}{L01}
\newcommand{\lecttitle}{Unit organisation}

\usepackage{amsmath, amssymb, graphicx}
\usepackage[]{algorithm2e}
\usepackage{pdfpages}
\usepackage[british]{babel}

\hypersetup{colorlinks,linkcolor=,urlcolor=blue}
\newenvironment{titledslide}[1]{\begin{frame}\frametitle{#1}}{\end{frame}}

\mode<presentation>{\setbeamercovered{transparent}}

\setbeamertemplate{sidebar right}{}
\setbeamertemplate{footline}{%
\hfill\usebeamertemplate***{navigation symbols}
\hspace{0.4cm}\lectnum: \insertframenumber{}/\inserttotalframenumber \hspace*{0.4cm}}

\author{James Cussens}

\title{COMS30035, Machine learning:\\ \vspace{5pt} \lecttitle}

\institute{School of Computer Science\\University of Bristol}

\begin{document}
%%%%%%%%%%%%%%%%%%%%%%%%%%%%%%%%%%%%%%%%%%%%%%%%%%%%%%%%%%%%%%%%%%%%%%

\begin{frame}
  \titlepage
\end{frame}

%%%%%%%%%%%%%%%%%%%%%%%%%%%%%%%%%%%%%%%%%%%%%%%%%%%%%%%%%%%%%%%%%%%%%%


%%%%%%%%%%%%%%%%%%%%%%%%%%%%%%%%%%%%%%%%%%%%%%%%%%%%%%%%%%%%%%%%%%%%%%
\begin{titledslide}{Unit github page}

  \begin{itemize}
  \item Most teaching materials and information are accessed via the
    unit's github page:
    \href{https://uob-coms30035.github.io/}{https://uob-coms30035.github.io/}
  \item This also gives you the unit schedule.
  \item There is a Blackboard page, which has quizzes and material
    that is unsuitable to make public.
  \end{itemize}
  
\end{titledslide}
%%%%%%%%%%%%%%%%%%%%%%%%%%%%%%%%%%%%%%%%%%%%%%%%%%%%%%%%%%%%%%%%%%%%%%
\begin{titledslide}{Teaching staff}

  \begin{itemize}
  \item The lecturers on this unit are:
    \begin{itemize}
    \item \href{https://jcussens.github.io/}{James Cussens}
    \item \href{https://zhang-xiyue.github.io/}{Xiyue Zhang}
    \item \href{https://weihonglee.github.io/}{Wei-Hong Li}
    \item \href{https://xiangli.ac.cn/}{Xiang Li} 
    \end{itemize}
  \item The following people are GTAs on the unit: Siddhant Bansal,
    Omar Emara, Kal Roberts, Jonathan Erskine, Enrique Crespo
    Fernandez and Zhiyuan Xu
  \item James Cussens gives most of the lectures.
  \item Both a lecturer and GTAs are present at drop-in sessions and labs.
  \end{itemize}
  
\end{titledslide}
%%%%%%%%%%%%%%%%%%%%%%%%%%%%%%%%%%%%%%%%%%%%%%%%%%%%%%%%%%%%%%%%%%%%%%
\begin{titledslide}{Teams}

  \begin{itemize}
  \item There is a Teams group ``COMS30035: Machine Learning (Teaching
    Unit) 2025/26 (TB-1, A)'' for this unit. It has 2 public
    channels:
    \begin{description}
    \item[General] Read-only for students. (We might use this for
      announcements, but we can also use Blackboard for that.)
    \item[Open] Students can post here. Feel free to use it as you
      wish; for example to ask questions.
    \end{description}
  \item There is a separate Teams channel ``COMS30083: Machine
    Learning 2025/26 (TB-1, A)''for students doing this unit as a
    MAJOR which is set up similarly and will be used for
    MAJOR-specific stuff (such as the coursework).
  \end{itemize}
  
\end{titledslide}
%%%%%%%%%%%%%%%%%%%%%%%%%%%%%%%%%%%%%%%%%%%%%%%%%%%%%%%%%%%%%%%%%%%%%% 
\begin{titledslide}{Teaching sessions}

  \begin{itemize}
  \item Apart from Week 1, lectures are 0900 Monday and 1500 Monday.
  \item There is one 3-hour lab each week starting at 1500 on Tuesday.
  \item Drop-in session on Thursdays (1700).
  \item During week 6 there is no teaching and during the coursework
    period (weeks 9-11) there is no teaching apart from coursework
    support sessions (1500-1700 Tuesdays).
  \end{itemize}
  
\end{titledslide}
%%%%%%%%%%%%%%%%%%%%%%%%%%%%%%%%%%%%%%%%%%%%%%%%%%%%%%%%%%%%%%%%%%%%%%
\begin{titledslide}{Textbooks}

  \begin{itemize}
  \item Bishop, C. M., Pattern recognition and machine learning
    (2006). Available for free \href{https://www.microsoft.com/en-us/research/people/cmbishop/prml-book/}{here}.	
  \item Murphy, K., Probabilistic Machine Learning: An Introduction
    (2022). This book is also freely available
    \href{https://probml.github.io/pml-book}{here}.
  \item Bishop is used more than Murphy.
  \item At the end of each set of lecture slides there will be a
    pointer stating where to find the relevant material in Bishop
    and/or Murphy.
  \item There will also (typically) be pointers to some problems taken
    from these textbooks.
  \item You can get help with these at drop-in sessions. It's fine to
    get help with a problem from an earlier week.
  \end{itemize}

\end{titledslide}
%%%%%%%%%%%%%%%%%%%%%%%%%%%%%%%%%%%%%%%%%%%%%%%%%%%%%%%%%%%%%%%%%%%%%%
\begin{titledslide}{Quizzes and problems}

  \begin{itemize}
  \item There are both Blackboard quizzes ('Unit Information and
    Resources' $\rightarrow$ `Quizzes' in \emph{Machine Learning
      (Teaching Unit) 2025} Blackboard site) and problems
    associated with the lecture material.
  \item Pointers to both are included at the end of the slides for
    the relevant lecture.
  \item These are \textbf{formative} not \textbf{summative}, i.e.\
    they do not count towards your mark for this unit.
  \end{itemize}
  
\end{titledslide}
%%%%%%%%%%%%%%%%%%%%%%%%%%%%%%%%%%%%%%%%%%%%%%%%%%%%%%%%%%%%%%%%%%%%%%
\begin{titledslide}{Drop-in sessions}

  \begin{itemize}
  \item You are not obliged to attend any drop-in sessions.
  \item You can get help with the quizzes and problems by coming to
    a drop-in session.
  \item But you can also ask us about anything at the drop-in sessions,
    e.g.\ further explanation of lecture material, pointers to
    additional reading, etc. 
  \end{itemize}
  
\end{titledslide}
%%%%%%%%%%%%%%%%%%%%%%%%%%%%%%%%%%%%%%%%%%%%%%%%%%%%%%%%%%%%%%%%%%%%%%
\begin{titledslide}{Assessment}

  \begin{itemize}
  \item If you are doing this unit as a MAJOR (COMS30083), then you
    are assessed by a mid-term in-class test in week 6 (30\%) and
    coursework (70\%) done during weeks 9-11.
  \item If you are doing this unit as a MINOR (COMS30081), then you
    are assessed by answering Machine Learning questions (equivalent
    to a 1-hour exam) in the \emph{Topics in Computer Science}
    December exam. This is a closed-book exam.
  \end{itemize}
  
\end{titledslide}
%%%%%%%%%%%%%%%%%%%%%%%%%%%%%%%%%%%%%%%%%%%%%%%%%%%%%%%%%%%%%%%%%%%%%%
\begin{titledslide}{Mid-term test}

  \begin{itemize}
  \item The mid-term test will be a pen-and-paper exam 1300-1400 on
    Wednesday 29 October. We will use rooms MVB 1.07, MVB 1.08 and QB 1.80.
  \item It will be closed book and you are not permitted to bring
    notes into the exam.
  \item Evidently only material from Weeks 1--5 will be examined.
  \end{itemize}
  
\end{titledslide}
%%%%%%%%%%%%%%%%%%%%%%%%%%%%%%%%%%%%%%%%%%%%%%%%%%%%%%%%%%%%%%%%%%%%%%
\begin{titledslide}{Revision materials}

  \begin{itemize}
  \item All last year's past assessments (MINOR exam, MAJOR in-class
    test and MAJOR coursework) are available at the Blackboard site:
    ('Unit Information and Resources' $\rightarrow$ `Previous
    assessments')
  \item In addition, there is a `mock' MINOR exam and a `mock' MAJOR in-class
    test, which was supplied to last year's students (in the absence
    of appropriate past exams).
  \item For the MINOR exams and MAJOR in-class tests there is a
    version with answers and a version without.
  \item For the coursework cohort-level feedback is provided
    (not the individual feedback of course!).
  \item The cohort-level feedback is a text file and, since BB seems
    unable to display it, you will have to download it to see it.
  \end{itemize}
  
\end{titledslide}
%%%%%%%%%%%%%%%%%%%%%%%%%%%%%%%%%%%%%%%%%%%%%%%%%%%%%%%%%%%%%%%%%%%%%%
\begin{titledslide}{Cribsheet}

  \begin{itemize}
  \item A `cribsheet' is available from the unit github page (in the
    Unit Materials section).
  \item This is a PDF which contains a list of things you need to
    know/understand.
  \item Occasionally, things you \textbf{don't need to know} are given.
  \end{itemize}
  
\end{titledslide}
%%%%%%%%%%%%%%%%%%%%%%%%%%%%%%%%%%%%%%%%%%%%%%%%%%%%%%%%%%%%%%%%%%%%%%
\begin{titledslide}{Labs}

  \begin{itemize}
  \item During lab sessions you do machine learning exercises.
  \item You can use either the machines in the room or your own
    machine.
  \item The machines in the room have the necessary ML software
    installed.
  \item You need to take certain steps to access that software: see
    the section on Lab Work on the
    \href{https://uob-coms30035.github.io/}{unit github page}.
  \item If you choose to use your own machine you will have to install
    the necessary software yourself. This is entirely doable and we're
    happy to attempt to help you if you hit installation problems, but
    we can't guarantee to be able to fix all such problems
    (particularly if you choose to use Windows).
  \item Labs are the most important part of the unit since there you
    actually \textbf{do} machine learning, rather than merely hear
    about it.
  \end{itemize}
  
\end{titledslide}
%%%%%%%%%%%%%%%%%%%%%%%%%%%%%%%%%%%%%%%%%%%%%%%%%%%%%%%%%%%%%%%%%%%%%%
\begin{titledslide}{The role of labs}

  \begin{itemize}
  \item The coursework will require you to apply the ML software we have
    been using in labs to machine learning problems. So doing lab work
    leads directly into doing coursework.
  \item Clearly, lab attendance is less crucial for those of you who
    will not be doing coursework.
  \item But, apart from the sheer joy of learning, I encourage
    students doing this unit as a MINOR to attend labs.
  \item Actively using, say, a Support Vector Machine method to learn
    a classifier, is more educational than passively having them
    explained to you.
  \end{itemize}
  
\end{titledslide}
%%%%%%%%%%%%%%%%%%%%%%%%%%%%%%%%%%%%%%%%%%%%%%%%%%%%%%%%%%%%%%%%%%%%%%
\begin{titledslide}{Lab materials}

  \begin{itemize}
  \item The materials for each lab are contained in a github repo:
    \href{https://github.com/uob-COMS30035/lab_sheets_public}{https://github.com/uob-COMS30035/lab\_sheets\_public}.
  \item For each lab you are presented with a Jupyter notebook which
    contains ML exercises for you to do. You complete each exercise by
    adding the necessary Python code to the notebook.
  \item There are links to these notebooks in the schedule on the
    unit's github page.
  \item Most labs require files (e.g.\ data, utility Python scripts)
    in addition to the notebook.
  \item The easiest way to ensure you have everything you need for
    each lab is to simply clone the git repo.
  \item (You can update your local git repo of the labs with a
    \texttt{git pull} before each lab, just to be sure you have
    any updates.)     
  \end{itemize}
  
\end{titledslide}
%%%%%%%%%%%%%%%%%%%%%%%%%%%%%%%%%%%%%%%%%%%%%%%%%%%%%%%%%%%%%%%%%%%%%%
\begin{titledslide}{Answers to formative material}

  \begin{itemize}
  \item Answers to lab exercises and any problems \textbf{not} from
    Bishop or Murphy will be released after the relevant lab.
  \item The problems selected from Bishop and Murphy have solutions
    available from the web sites associated with the book.
  \item Quiz answers are not `released' (the are multiple-choice). The
    quizzes are set up \textbf{not} to give you the correct answer if
    you give an incorrect answer. This is deliberate!
  \end{itemize}
  
\end{titledslide}
%%%%%%%%%%%%%%%%%%%%%%%%%%%%%%%%%%%%%%%%%%%%%%%%%%%%%%%%%%%%%%%%%%%%%%
\begin{titledslide}{Python}

  \begin{itemize}
  \item All your coding on this unit will be done in Python.
  \item This code will call ML library methods supplied by packages
    such as scikit-learn, pytorch and PyMC.
  \item (Note that Python is mostly used a wrapper; the native code
    called by Python was typically compiled from C/C++/FORTRAN source.) 
  \item Your code will either be a regular script or in a JupyterLab
    notebook.
  \item Python coding, per se, is not taught on this unit. 
  \end{itemize}
  
\end{titledslide}

\end{document}
