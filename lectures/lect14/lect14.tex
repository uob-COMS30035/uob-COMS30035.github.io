\documentclass[10pt]{beamer}

\newcommand{\lectnum}{L14}
\newcommand{\lecttitle}{Sequential Data (LDS)}

\usepackage{amsmath, amssymb, graphicx}
\usepackage[]{algorithm2e}
\usepackage{pdfpages}
\usepackage[british]{babel}

\hypersetup{colorlinks,linkcolor=,urlcolor=blue}
\newenvironment{titledslide}[1]{\begin{frame}\frametitle{#1}}{\end{frame}}

\mode<presentation>{\setbeamercovered{transparent}}

\setbeamertemplate{sidebar right}{}
\setbeamertemplate{footline}{%
\hfill\usebeamertemplate***{navigation symbols}
\hspace{0.4cm}\lectnum: \insertframenumber{}/\inserttotalframenumber \hspace*{0.4cm}}

\author{James Cussens}

\title{COMS30035, Machine learning:\\ \vspace{5pt} \lecttitle}

\institute{School of Computer Science\\University of Bristol}

\begin{document}
%%%%%%%%%%%%%%%%%%%%%%%%%%%%%%%%%%%%%%%%%%%%%%%%%%%%%%%%%%%%%%%%%%%%%%

\begin{frame}
  \titlepage
\end{frame}

%%%%%%%%%%%%%%%%%%%%%%%%%%%%%%%%%%%%%%%%%%%%%%%%%%%%%%%%%%%%%%%%%%%%%%



%%%%%%%%%%%%%%%%%%%%%%%%%%%%%%%%%%%%%%%%%%%%%%%%%%%%%%%%%%%%%%%%%%%%%%
\begin{titledslide}{Acknowledgement}

  \begin{itemize}
  \item These slides are adapted from ones originally created by Edwin Simpson. 
  \end{itemize}
  
\end{titledslide}


%%%%%%%%%%%%%%%%%%%%%%%%%%%%%%%%%%%%%%%%%%%%%%%%%%%%%%%%%%%%%%%%%%%%%%

\begin{frame}[fragile]
\frametitle{Agenda}   % expectation: 33 slides + title and agenda, 11 slides per chunk
\begin{itemize}
\item  \textcolor{gray}{Markov Models}
\item \textcolor{gray}{Hidden Markov Models}
\item \textcolor{gray}{EM for HMMs}
\item Linear Dynamical Systems
%\item \textcolor{gray}{Bayesian Timeseries Modelling with Gaussian Processes}  
\end{itemize}
\end{frame}


\begin{frame}
\frametitle{From HMM to LDS}
\begin{itemize}
\item HMM assumes discrete latent states.
\item Linear dynamical systems (LDS) assume states have continuous values.
\item Both have the same graphical model:
\end{itemize}
\begin{tikzpicture}
\node[draw=none]   (0) at (-1,-2) {};
\node[draw=none]   (1) at (0,-2) {$z_1$};
\draw[style={, ->, >=stealth'}] (0) edge node [right] {} (1);
\draw (0,-2)circle (0.3cm);

\node[draw=none] (2) at (1,-2) {$z_2$};
\draw[style={, ->, >=stealth'}] (1) edge node [right] {} (2);
\draw (1,-2) circle (0.3cm);

\node[draw=none] (3) at (2,-2) {$z_3$};
\draw[style={, ->, >=stealth'}](2) edge node [right] {} (3);
\draw (2,-2) circle (0.3cm);

\node[draw=none] (4) at (3,-2) {$z_4$};
\draw[style={, ->, >=stealth'}](3) edge node [right] {} (4);
\draw (3,-2) circle (0.3cm);

\node[draw=none]   (9) at (4,-2) {};
\draw[style={, ->, >=stealth'}](4) edge node [right] {} (9);

\node[draw=none]   (5) at (0,-4) {$x_1$};
\draw[style={, ->, >=stealth'}] (1) edge node [right] {} (5);
\draw (0,-4)circle (0.3cm);

\node[draw=none] (6) at (1,-4) {$x_2$};
\draw[style={, ->, >=stealth'}] (2) edge node [right] {} (6);
\draw (1,-4) circle (0.3cm);

\node[draw=none] (7) at (2,-4) {$x_3$};
\draw[style={, ->, >=stealth'}](3) edge node [right] {} (7);
\draw (2,-4) circle (0.3cm);

\node[draw=none] (8) at (3,-4) {$x_4$};
\draw[style={, ->, >=stealth'}](4) edge node [right] {} (8);
\draw (3,-4) circle (0.3cm);
\end{tikzpicture}
\begin{itemize}
\item Inference has the same form as for an HMM, but when marginalising $\bs z_{n-1}$ and 
$\bs z_{n+1}$, we take integrals instead of sums.
\end{itemize}

\end{frame}


\begin{frame}
\frametitle{Motivations for LDS}
\begin{itemize}
\item Noisy sensors: inferring the true sequence of states from observations with Gaussian noise.
\item Tracking: predicting the next movement and tracing the path from noisy observations.
\end{itemize}
% plot of an example
\end{frame}


\begin{frame}
\frametitle{Transition and Emission Distributions for LDS}
\begin{itemize}
\item $p(\bs z_1) = \mathcal{N}(\bs z_1 | \bs\mu_0, \bs V_0)$;
\item $p(\bs z_n | \bs z_{n-1}) = \mathcal{N}(\bs z_n | \bs A \bs z_{n-1}, \bs\Gamma)$;
\item $p(\bs x_n | \bs z_n) = \mathcal{N}(\bs x_n | \bs C \bs z_n, \bs\Sigma)$.
%\item You can alternatively view the observations as adding Gaussian noise to the states.
\item Note that the means of both distributions are \emph{linear} functions of the latent states.
\item This choice of distributions ensures that the posteriors are also Gaussians with updated parameters
\item This means that $\mathcal{O}(N)$ inference can still be performed using the sum-product algorithm.
 
\end{itemize}
% plot of an example
\end{frame}


\begin{frame}
\frametitle{Inference for an LDS}
\begin{itemize}
\item \emph{Kalman filter} = forward pass of sum-product for LDS.
\item \emph{Kalman smoother} = backward pass of sum-product for LDS.
\item No need for an analogue of Viterbi: the most likely sequence is given by the individually most probable states,
so we get this from the Kalman equations.
 
\end{itemize}
% plot of an example
\end{frame}


\begin{frame}
\frametitle{Forward Inference (Kalman Filter) for an LDS}
\begin{itemize}
%\item Normalized "messages": 
%$\hat{\alpha}(\bs z_n) = \mathcal{N}(\bs z_n | \bs \mu_n, \bs V_n) = p(\bs z_n | \bs x_1,...,\bs x_N)$
\item \begin{align}
{\alpha}(\bs z_n) & = \mathcal{N}(\bs x_n | \bs C \bs z_n, \bs \Sigma) \int \mathcal{N}(\bs z_n | 
\bs A \bs z_{n-1}, \bs\Gamma) \alpha(\bs z_{n-1})  \mathrm{d} \bs z_{n-1} %= p(\bs z_n | \bs x_1,...,\bs x_N)
%\nonumber \\
%& = 
\end{align}
\item Normalising results in a Gaussian-distributed variable, whose parameters can be computed
efficiently:
$\hat{\alpha}(\bs z_n) = p(\bs z_n | \bs x_1, ..., \bs x_n) = \mathcal{N}(\bs z_n | \bs \mu_n, \bs V_n )$, where
\begin{itemize}
\item $\bs\mu_n$ is a function of $\bs\mu_{n-1}$,  $\bs x_n$, $\bs A$ and $\bs C$.
\item  $\bs V_n$ is a function of $\bs V_{n-1}$,  $\bs\Sigma$, $\bs A$, $\bs\Gamma$ and $\bs C$.
\end{itemize}
\item We can view each forward step as predicting $\bs z_n$ based on the distribution over $\bs z_{n-1}$, then correcting
that prediction given the new observation $\bs x_n$.
% note that as we predict forward in time, we increase uncertainty in z_n to represent the possible state transitions
\item For details, see Bishop (2006), Section 13.3.1
\end{itemize}
% plot of an example
\end{frame}


\begin{frame}
\frametitle{Backward Inference (Kalman Smoother) for an LDS}
\begin{itemize}
%\item Normalized "messages": 
%$\hat{\alpha}(\bs z_n) = \mathcal{N}(\bs z_n | \bs \mu_n, \bs V_n) = p(\bs z_n | \bs x_1,...,\bs x_N)$
\item Backward pass also follows that of the HMM: messages are passed from the final
state to the start of the sequence.
\item The backward messages contain information about future states that affects the posterior distribution at each step $n$.
\item Since the transition and emission probabilities are all Gaussian, the posterior \emph{responsibilities} are also Gaussian, as are the \emph{state pair} expectations.
\item For details, see Bishop (2006), Section 13.3.1
\end{itemize}
% plot of an example
\end{frame}


\begin{frame}
\frametitle{Learning the Parameters of LDS}
\begin{itemize}
%\item Normalized "messages": 
%$\hat{\alpha}(\bs z_n) = \mathcal{N}(\bs z_n | \bs \mu_n, \bs V_n) = p(\bs z_n | \bs x_1,...,\bs x_N)$
\item Kalman filter/smoother are analogous to the forward-backward algorithm for HMMs.
\item Remember that this algorithm is used for the \emph{E step} of EM.
\item The parameters are optimised in the \emph{M step} as before, by using the responsibilities
$\mathbb{E}[\bs z_n]$, $\mathbb{E}[\bs z_n \bs z_n^T]$ and state pair expectations $\mathbb{E}[\bs z_n \bs z_{n-1}^T]$.
\item For details, see Bishop (2006), Section 13.3.2
\end{itemize}
% plot of an example
\end{frame}

\begin{frame}
\frametitle{Now do the quiz!}

Please do the quiz for this lecture on Blackboard.
\end{frame}


\end{document}
