\documentclass[10pt]{beamer}

\newcommand{\lectnum}{L10}
\newcommand{\lecttitle}{Markov Chain Monte Carlo}

\usepackage{amsmath, amssymb, graphicx}
\usepackage[]{algorithm2e}
\usepackage{pdfpages}
\usepackage[british]{babel}

\hypersetup{colorlinks,linkcolor=,urlcolor=blue}
\newenvironment{titledslide}[1]{\begin{frame}\frametitle{#1}}{\end{frame}}

\mode<presentation>{\setbeamercovered{transparent}}

\setbeamertemplate{sidebar right}{}
\setbeamertemplate{footline}{%
\hfill\usebeamertemplate***{navigation symbols}
\hspace{0.4cm}\lectnum: \insertframenumber{}/\inserttotalframenumber \hspace*{0.4cm}}

\author{James Cussens}

\title{COMS30035, Machine learning:\\ \vspace{5pt} \lecttitle}

\institute{School of Computer Science\\University of Bristol}

\begin{document}
%%%%%%%%%%%%%%%%%%%%%%%%%%%%%%%%%%%%%%%%%%%%%%%%%%%%%%%%%%%%%%%%%%%%%%

\begin{frame}
  \titlepage
\end{frame}

%%%%%%%%%%%%%%%%%%%%%%%%%%%%%%%%%%%%%%%%%%%%%%%%%%%%%%%%%%%%%%%%%%%%%%

\usetikzlibrary{bayesnet}

\newcommand{\ci}{\ensuremath{\perp}}
\newcommand{\params}{\ensuremath{\mathbf{w}}}
\newcommand{\dualparams}{\ensuremath{\mathbf{a}}}
\newcommand{\designm}{\ensuremath{{\bm \Phi}}}
\newcommand{\xvec}{\ensuremath{\mathbf{x}}}
\newcommand{\xvecn}{\ensuremath{\mathbf{x}_{n}}}
\newcommand{\feat}{\ensuremath{\phi}}
\newcommand{\gram}{\ensuremath{\mathbf{K}}}

%%%%%%%%%%%%%%%%%%%%%%%%%%%%%%%%%%%%%%%%%%%%%%%%%%%%%%%%%%%%%%%%%%%%%%
\begin{titledslide}{The Bayesian approach}

  \begin{itemize}
  \item Conceptually the Bayesian approach is easy: the goal is to
    compute the posterior distribution $P(\theta|D=d)$ where $\theta$
    is the parameter vector and $d$ is the observed value of the data.
  \item We choose a prior $P(\theta)$ and assume a particular
    likelihood $P(D|\theta)$ and then Bayes theorem gives us
    $P(\theta|D=d) \propto P(\theta)P(D=d|\theta)$.
  \item If we choose a \emph{conjugate prior} for $P(\theta)$, then
    representing and computing $P(\theta|D=d)$ is easy.
  \end{itemize}
\end{titledslide}
%%%%%%%%%%%%%%%%%%%%%%%%%%%%%%%%%%%%%%%%%%%%%%%%%%%%%%%%%%%%%%%%%%%%%% 
\begin{titledslide}{Problems for the Bayesian approach}

  \begin{itemize}
  \item ``For most probabilistic models of practical interest, exact
    inference is intractable, and so we have to resort to some form of
    approximation.'' \cite[p. 523]{bishop06:_patter_recog_machin_learn}.
  \item We want to be able to just construct whatever joint
    distribution $P(\theta,D)$ we think best models the
    data-generating process and then compute $P(\theta|D=d)$.
  \item However, with this flexibility there is a price: we may not
    even be able to represent $P(\theta|D=d)$ easily, let alone
    compute it.\pause
  \item The solution is to give up on getting $P(\theta|D=d)$ exactly
    and instead draw samples (of $\theta$) from $P(\theta|D=d)$ which
    will allow us to approximately compute any posterior quantities,
    e.g.\ the mean of $P(\theta|D=d)$.
  \end{itemize}
\end{titledslide}
%%%%%%%%%%%%%%%%%%%%%%%%%%%%%%%%%%%%%%%%%%%%%%%%%%%%%%%%%%%%%%%%%%%%%% 
\begin{titledslide}{Univariate sampling}

  \begin{itemize}
  \item We will assume throughout that we have some mechanism for
    sampling from any \emph{univariate} distribution.
  \item There are functions for sampling from a bunch of different
    distributions in Python's
    \href{https://docs.python.org/3/library/random.html}{random
      module}. Also, to sample from a Gaussian you can use
    \texttt{numpy.random.normal}.
  \item If a multivariate distribution is described by a Bayesian
    network then we can use \emph{ancestral sampling} to sample a
    joint instantiation of the variables.
  \end{itemize}

\end{titledslide}
%%%%%%%%%%%%%%%%%%%%%%%%%%%%%%%%%%%%%%%%%%%%%%%%%%%%%%%%%%%%%%%%%%%%%%
\begin{titledslide}{Ancestral sampling}

  \[
p(A,B,C,D,E)=p(A)p(B)p(C|A,B)p(D|C)p(E|B,C)
\]

  
   \begin{tikzpicture}%[dgraph]
 \node[latent] (x1) at (0,2.1) {$A$};
 \node[latent] (x2) at (2,2.1) {$B$};
 \node[latent] (x3) at (1,1) {$C$};
 \node[latent] (x4) at (-0.2,0.1) {$D$};
 \node[latent] (x5) at (2.5,-0.1) {$E$};
 \edge{x1}  {x3};
 \edge{x2} {x5};
 \edge{x2} {x3};
 \edge{x3} {x4};
 \edge{x3}  {x5};
 \end{tikzpicture}

 \begin{itemize}
 \item Just ensure that we sample values for all parents of a node
   before we sample a value for that node (this is always possible due
   to acyclicity).
 \item So to sample from $p(A,B,C,D,E)$ we first sample values for $A$
   and $B$, suppose we get the values $A=0,B=1$. We then sample a
   value for $C$ from the conditional distribution $P(C|A=0,B=1)$, and
   so on. \cite[\S8.1.2]{bishop06:_patter_recog_machin_learn}.
  \end{itemize}
\end{titledslide}
%%%%%%%%%%%%%%%%%%%%%%%%%%%%%%%%%%%%%%%%%%%%%%%%%%%%%%%%%%%%%%%%%%%%%% 
\begin{titledslide}{Sampling from marginal and conditional distributions}

  \[
    p(A,B,C,D,E)=p(A)p(B)p(C|A,B)p(D|C)p(E|B,C)
  \]
  
  
  \begin{itemize}
  \item We can approximate any marginal distribution (say, $P(B,E)$)
    by sampling full joint instantiations (by e.g.\ ancestral
    sampling) and then only keeping the values of the variables in
    the marginal.
  \item We can use \emph{rejection sampling} to sample from
    conditional distributions.
  \item For example, to sample from $P(B,D|E=1)$ we sample from the
    marginal distribution $P(B,D,E)$ and throw away those samples
    where $E\neq 1$.
  \item Rejection sampling is typically inefficient.
  \end{itemize}
   
\end{titledslide}
%%%%%%%%%%%%%%%%%%%%%%%%%%%%%%%%%%%%%%%%%%%%%%%%%%%%%%%%%%%%%%%%%%%%%%
\begin{titledslide}{Approximating expectations}

  \begin{itemize}
  \item Often we want to compute expected values with respect to some
    posterior distribution \cite[p. 524]{bishop06:_patter_recog_machin_learn}.
  \end{itemize}

  \begin{equation}
    \label{eq:exp}
    E[f] = \int f(\mathbf{z})p(\mathbf{z}) d\mathbf{z}
  \end{equation}

  \begin{itemize}
  \item   If we draw independent samples $\mathbf{z}^{(l)}$, $l = 1, \dots, L$
  from $p(\mathbf{z})$ then we can approximate $E[f]$ as follows:
  \end{itemize}

  \begin{equation}
    \label{eq:approxexp}
    \hat{f} = \frac{1}{L}\sum_{l=1}^{L}f(\mathbf{z}^{(l)})
  \end{equation}
  
\end{titledslide}
%%%%%%%%%%%%%%%%%%%%%%%%%%%%%%%%%%%%%%%%%%%%%%%%%%%%%%%%%%%%%%%%%%%%%%
\begin{titledslide}{Markov chain Monte Carlo}

  \begin{itemize}
  \item If we can sample from a distribution then we have a simple way to
    compute approximate values. But what if we cannot? \pause
  \item If we can sample from \emph{a sequence of distributions} which
    eventually reaches (or gets very close to) the desired
    distribution, then we can adopt the following strategy:
    \begin{enumerate}
    \item Draw a sample from each distribution in this sequence.
    \item Only keep the samples once we get `close enough' to the
      desired distribution.
    \end{enumerate}
  \item This is the approach of Markov chain Monte Carlo (MCMC).
  \end{itemize}
  
\end{titledslide}
%%%%%%%%%%%%%%%%%%%%%%%%%%%%%%%%%%%%%%%%%%%%%%%%%%%%%%%%%%%%%%%%%%%%%%
\begin{titledslide}{Markov chains}

  
  ``A first-order Markov chain is defined to be a series of random
  variables $\mathbf{z}^{(1)}, \dots, \mathbf{z}^{(M)}$ such that the
  following conditional independence property holds for
  $m \in \{1, \dots, M-1\}$''
  \cite[p. 539]{bishop06:_patter_recog_machin_learn}.
\begin{equation}
p(\mathbf{z}^{(m+1)} |\mathbf{z}^{(1)} , . . . , \mathbf{z}^{(m)} ) = p(\mathbf{z}^{(m+1)} |\mathbf{z}^{(m)} )  
\end{equation}

\begin{itemize}
\item $\mathbf{z}^{(m)}$ often represents (or can be imagined to
  represent) the $m$th state of some dynamic system so that
  $p(\mathbf{z}^{(m+1)} |\mathbf{z}^{(m)} )$ is a \emph{state
    transition probability}. 
\item If $p(\mathbf{z}^{(m+1)} |\mathbf{z}^{(m)} )$ is the same for all $m$ then the chain is \emph{homogeneous}.
\item (We also need an \emph{initial distribution} $p(\mathbf{z}^{(1)})$.)
\item Here's the Bayesian network representation of a Markov chain where $M=4$.
\end{itemize}

\begin{center}
\begin{tikzpicture}
  \node [latent] (x1) {$\mathbf{z}_1$};
  \node[latent,right=of x1] (x2) {$\mathbf{z}_2$};
  \node[latent,right=of x2] (x3) {$\mathbf{z}_3$};
  \node[latent,right=of x3] (x4) {$\mathbf{z}_4$};
  \edge{x1} {x2};
  \edge{x2} {x3};
  \edge{x3} {x4};
\end{tikzpicture}
\end{center}

\begin{itemize}
\item Sampling from a Markov chain is easy: it's just a special case
  of ancestral sampling.
\end{itemize}
\end{titledslide}
%%%%%%%%%%%%%%%%%%%%%%%%%%%%%%%%%%%%%%%%%%%%%%%%%%%%%%%%%%%%%%%%%%%%%%
\begin{titledslide}{Markov chain Monte Carlo}

\begin{center}
\begin{tikzpicture}
  \node [latent] (x1) {$\mathbf{z}_1$};
  \node[latent,right=of x1] (x2) {$\mathbf{z}_2$};
  \node[latent,right=of x2] (x3) {$\mathbf{z}_3$};
  \node[latent,right=of x3] (x4) {$\mathbf{z}_4$};
  \edge{x1} {x2};
  \edge{x2} {x3};
  \edge{x3} {x4};
\end{tikzpicture}
\end{center}

\begin{itemize}
\item A Markov chain defines a sequence of marginal distributions; for
  the BN above these are $P(\mathbf{z}_{1})$, $P(\mathbf{z}_{2})$, $P(\mathbf{z}_{3})$ and
  $P(\mathbf{z}_{4})$.
\item The goal of MCMC is to design a Markov chain so that this
  sequence of marginal distributions converges on the distribution we
  want. 
\item Then we can just sample from the Markov chain and only keep the
  sampled values of the `later' random variables.
\item The sampled values we draw are \textbf{not} independent (which
  reduces the quality of the approximations we end up with), but this
  is a price we have to pay.
\end{itemize}

\end{titledslide}
%%%%%%%%%%%%%%%%%%%%%%%%%%%%%%%%%%%%%%%%%%%%%%%%%%%%%%%%%%%%%%%%%%%%%%
\begin{titledslide}{How to get MCMC to work?}

  \begin{itemize}
  \item We have a clear goal:
    \textbf{given} a target probability distribution $p(\mathbf{z})$,
    \textbf{construct} a Markov chain
    $\mathbf{z}^{(1)}, \dots, \mathbf{z}^{(i)} \dots$ such that
    $\lim_{i \rightarrow \infty} p(\mathbf{z}^{(i)}) = p(\mathbf{z})$.
  \item (For Bayesian machine learning the target distribution will be
    $P(\theta|D=d)$, the posterior distribution of the model
    parameters given the observed data.)
  \item One solution to this is the \emph{Metropolis-Hastings}
    algorithm.
  \end{itemize}
\end{titledslide}
%%%%%%%%%%%%%%%%%%%%%%%%%%%%%%%%%%%%%%%%%%%%%%%%%%%%%%%%%%%%%%%%%%%%%%
\begin{titledslide}{The Metropolis-Hastings (MH) algorithm}

  \begin{itemize}
%  \item Let $p(\mathbf{z})$ be the \emph{target distribution}.
  \item We define a single transition probability distribution for a
    homogeneous Markov chain.
  \item Let the current state be $\mathbf{z}^{(\tau)}$. When using the
    MH algorithm sampling the
    next state happens in two stages:
    \begin{enumerate}
    \item We generate a value $\mathbf{z}^{*}$ by sampling from a
      \emph{proposal distribution}
      $q(\mathbf{z}|\mathbf{z}^{(\tau)})$.
    \item We then accept $\mathbf{z}^{*}$ as the new state with a
      certain \emph{acceptance probability} in which case
      $\mathbf{z}^{(\tau+1)}= \mathbf{z}^{*}$.  If we don't accept
      $\mathbf{z}^{*}$ then we `stay where we are', so that
      $\mathbf{z}^{(\tau+1)}= \mathbf{z}^{(\tau)}$.
    \end{enumerate}
  \end{itemize}

\end{titledslide}
%%%%%%%%%%%%%%%%%%%%%%%%%%%%%%%%%%%%%%%%%%%%%%%%%%%%%%%%%%%%%%%%%%%%%%
\begin{titledslide}{The Metropolis-Hastings acceptance probability}

  Let $p(\mathbf{z})$ be the \emph{target distribution}.  The
  acceptance probability is:
  \cite[p. 541]{bishop06:_patter_recog_machin_learn}.

  \begin{equation}
    \label{eq:acceptance}
    A(\mathbf{z}^{*},\mathbf{z}^{(\tau)}) =
    \min \left(
      1, \frac{p(\mathbf{z}^{*})q(\mathbf{z}^{(\tau)}|\mathbf{z}^{*})}
      {p(\mathbf{z}^{(\tau)})q(\mathbf{z}^{*}|\mathbf{z}^{(\tau)})} \right)
  \end{equation}

  \begin{itemize}
  \item   If $p(\mathbf{z}) = \tilde{p}(\mathbf{z})/Z$ then we have
  $p(\mathbf{z}^{*})/p(\mathbf{z}^{(\tau)}) =
    \tilde{p}(\mathbf{z}^{*})/\tilde{p}(\mathbf{z}^{(\tau)})$, so we
    only need $p$ up to normalisation. This is a big win!
  \item If the proposal distribution is symmetric then the `$q$' terms
    cancel out: a special case known as the \emph{Metropolis
      algorithm}.
  \item Note that for the Metropolis algorithm if
    $p(\mathbf{z}^{*}) \geq p(\mathbf{z}^{(\tau)})$ then we always
    accept and `move' to $\mathbf{z}^{*}$.
  \end{itemize}

  
\end{titledslide}
%%%%%%%%%%%%%%%%%%%%%%%%%%%%%%%%%%%%%%%%%%%%%%%%%%%%%%%%%%%%%%%%%%%%%%
\begin{titledslide}{Does Metropolis-Hastings (always) work?}.

  \begin{itemize}
  \item It can be shown
    \cite[p. 541]{bishop06:_patter_recog_machin_learn} that the target
    distribution is an \emph{invariant distribution of the Markov
      chain}: if the sequence of distributions $p(\mathbf{z}^{(i)})$
    reaches the target distribution then it stays there.
  \item Also, typically the Markov chain does converge to the target
    distribution.
  \item The \emph{rate} at which we converge to the target
    distribution is greatly influenced by the choice of proposal
    distribution.
  \item Let's look at \cite[Fig 12.1]{pml2book}.
  \end{itemize}


\end{titledslide}
%%%%%%%%%%%%%%%%%%%%%%%%%%%%%%%%%%%%%%%%%%%%%%%%%%%%%%%%%%%%%%%%%%%%%%
\begin{titledslide}{MCMC in practice}

  \begin{itemize}
  \item Straightforward Metropolis-Hastings is not the state-of-the-art
    in MCMC.
  \item \emph{Probabilistic programming} systems like PyMC by default
    use more sophisticated MCMC algorithms (to avoid getting stuck).
  \item From
    \href{https://www.pymc.io/projects/docs/en/latest/learn/core_notebooks/pymc_overview.html}{the
      PyMC intro overview}: ``Probabilistic programming (PP) allows
    flexible specification of Bayesian statistical models in
    code. PyMC is a PP framework with an intuitive and readable, yet
    powerful, syntax that is close to the natural syntax statisticians
    use to describe models. It features next-generation Markov chain
    Monte Carlo (MCMC) sampling algorithms such as the No-U-Turn
    Sampler''
  \item When using MCMC we (1) throw away early samples (`burn-in')
    and (2) `run independent chains' to check for convergence.
  \item PyMC uses $\hat{R}$ (\texttt{r\_hat}) to check for
    convergence; this value should be close to 1.
  \end{itemize}

  
\end{titledslide}
%%%%%%%%%%%%%%%%%%%%%%%%%%%%%%%%%%%%%%%%%%%%%%%%%%%%%%%%%%%%%%%%%%%%%%
\begin{titledslide}{Let's do some Bayesian machine learning with PyMC!}

  \begin{itemize}
  \item I've found the easiest way to get the introductory Jupyter notebooks
    mentioned in the PyMC website is to clone the PyMC github repo.
  \item You can then find them in \texttt{pymc/docs/source/learn/core\_notebooks}
  \end{itemize}
  
\end{titledslide}
%%%%%%%%%%%%%%%%%%%%%%%%%%%%%%%%%%%%%%%%%%%%%%%%%%%%%%%%%%%%%%%%%%%%%%
\begin{titledslide}{Reading}

  \begin{itemize}
  \item Bishop \S11.1.2.
  \item Bishop \S11.2.
  \item Murphy Book 2 \cite{pml2book}: \S12.1--\S12.2 (more detailed
    than you need).
  \end{itemize}
  
  
\end{titledslide}
%%%%%%%%%%%%%%%%%%%%%%%%%%%%%%%%%%%%%%%%%%%%%%%%%%%%%%%%%%%%%%%%%%%%%%
\begin{titledslide}{Problems and quizzes}

  \begin{itemize}
  \item No problems.
  \item Quizzes:
    \begin{itemize}
    \item Week~4: Bayesian Machine Learning 
    \item Week~4: Sampling and Markov Chains
    \item Week~4: MCMC/Metropolis-Hastings
    \end{itemize}
  \end{itemize}
  
\end{titledslide}
%%%%%%%%%%%%%%%%%%%%%%%%%%%%%%%%%%%%%%%%%%%%%%%%%%%%%%%%%%%%%%%%%%%%%% 
\bibliographystyle{alpha}
\bibliography{../ml}

\end{document}
%%%%%%%%%%%%%%%%%%%%%%%%%%%%%%%%%%%%%%%%%%%%%%%%%%%%%%%%%%%%%%%%%%%%%% 
