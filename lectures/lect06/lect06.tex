\documentclass[10pt]{beamer}

\newcommand{\lectnum}{L06}
\newcommand{\lecttitle}{Training Neural Networks}

\usepackage{amsmath, amssymb, graphicx}
\usepackage[]{algorithm2e}
\usepackage{pdfpages}
\usepackage[british]{babel}

\hypersetup{colorlinks,linkcolor=,urlcolor=blue}
\newenvironment{titledslide}[1]{\begin{frame}\frametitle{#1}}{\end{frame}}

\mode<presentation>{\setbeamercovered{transparent}}

\setbeamertemplate{sidebar right}{}
\setbeamertemplate{footline}{%
\hfill\usebeamertemplate***{navigation symbols}
\hspace{0.4cm}\lectnum: \insertframenumber{}/\inserttotalframenumber \hspace*{0.4cm}}

\author{James Cussens}

\title{COMS30035, Machine learning:\\ \vspace{5pt} \lecttitle}

\institute{School of Computer Science\\University of Bristol}

\begin{document}
%%%%%%%%%%%%%%%%%%%%%%%%%%%%%%%%%%%%%%%%%%%%%%%%%%%%%%%%%%%%%%%%%%%%%%

\begin{frame}
  \titlepage
\end{frame}

%%%%%%%%%%%%%%%%%%%%%%%%%%%%%%%%%%%%%%%%%%%%%%%%%%%%%%%%%%%%%%%%%%%%%%

%%%%%%%%%%%%%%%%%%%%%%%%%%%%%%%%%%%%%%%%%%%%%%%%%%%%%%%%%%%%%%%%%%%%%%
\begin{titledslide}{Problems and solutions when training neural networks}

  \begin{itemize}
  \item Once we have computed the gradient (at the current value of
    the weights) \dots
  \item \dots we can just send that gradient to an off-the-shelf
    gradient based optimiser to reduce error (on the training set).
  \item That can work fine, but not always.
  \item In this lecture we look at various problems that may arise and
    how they are dealt with.
  \end{itemize}
  
\end{titledslide}
%%%%%%%%%%%%%%%%%%%%%%%%%%%%%%%%%%%%%%%%%%%%%%%%%%%%%%%%%%%%%%%%%%%%%%
\begin{titledslide}{Lecture content}

  \begin{itemize}
  \item This lecture is nothing more than a summary of (some of) the
    material that can be found in Chapter~13 of Kevin Murphy's book
    \cite{pml1Book}
  \item We'll be switching from these slides to look at some of the
    figures in that Chapter occasionally.
  \end{itemize}
  
\end{titledslide}
%%%%%%%%%%%%%%%%%%%%%%%%%%%%%%%%%%%%%%%%%%%%%%%%%%%%%%%%%%%%%%%%%%%%%%
\begin{titledslide}{Vanishing and exploding gradients}

  \begin{itemize}
  \item When training deep models (ones with many layers) the gradient
    can become very small or very large.
  \item These are the \emph{vanishing gradient problem} and
    \emph{exploding gradient problem} respectively.
  \end{itemize}
  
\end{titledslide}
%%%%%%%%%%%%%%%%%%%%%%%%%%%%%%%%%%%%%%%%%%%%%%%%%%%%%%%%%%%%%%%%%%%%%%
\begin{titledslide}{Gradient clipping}

  \begin{itemize}
  \item \emph{Gradient clipping} can be used to cap the magnitude of
    the gradient.
  \item The original gradient $\mathbf{g}$ and the clipped gradient
    $\mathbf{g}'$ are related as follows:
  \end{itemize}

  \[
    \mathbf{g}' = \min \left(1, \frac{c}{||\mathbf{g}||}\right)\mathbf{g}
  \]

  \begin{itemize}
  \item   Note that $\mathbf{g}'$ points in the same direction as $\mathbf{g}$.
  \end{itemize}

\end{titledslide}
%%%%%%%%%%%%%%%%%%%%%%%%%%%%%%%%%%%%%%%%%%%%%%%%%%%%%%%%%%%%%%%%%%%%%%
\begin{titledslide}{Non-saturating activation functions for deep learning}

  \begin{itemize}
  \item The sigmoid activation function, for example, \emph{saturates}
    at large negative or large positive values.
  \item So its gradient there will be 0 preventing gradient descent
    from reaching good weights.
  \item A solution to this is simply to choose activation functions
    that don't saturate.
  \item Let's look at \cite[Fig~13.4]{pml1Book}.
  \item ReLU and its variants are popular choices. But not that plain
    ReLU has zero gradient for negative values so we might end up with
    the `dead ReLU' problem.
  \item Note also that ReLU is not a differentiable function at 0, but we
    can just impose a derivative value of 0 there.
  \end{itemize}
  
\end{titledslide}
%%%%%%%%%%%%%%%%%%%%%%%%%%%%%%%%%%%%%%%%%%%%%%%%%%%%%%%%%%%%%%%%%%%%%%
\begin{titledslide}{Using residual connections}

  \begin{itemize}
  \item In a \emph{residual network} each layer has the form of a
    \emph{residual layer} defined as:
  \end{itemize}

  \[
    {\cal F}'_{l}(\mathbf{x}) = {\cal F}_{l}(\mathbf{x}) + \mathbf{x} 
  \]

  \begin{itemize}
  \item where ${\cal F}_{l}(\mathbf{x})$ is the standard (linear then
    activation function) mapping.
  \item The weights defining ${\cal F}_{l}(\mathbf{x})$ learn what
    needs to be \textbf{added to} the input $\mathbf{x}$.
  \item This provides a `short cut' for gradients to flow directly
    from the output layer to any previous layer.
  \item Let's look at \cite[Fig~13.15]{pml1Book}.
  \end{itemize}
  
\end{titledslide}
%%%%%%%%%%%%%%%%%%%%%%%%%%%%%%%%%%%%%%%%%%%%%%%%%%%%%%%%%%%%%%%%%%%%%%
\begin{titledslide}{Parameter initialisation}

  \begin{itemize}
  \item Neural network training is non-convex, \dots
  \item \dots i.e.\ the error `landscape' has many `valleys' and standard
    gradient descent just ends up at the bottom of whichever valley it
    happens to start in.
  \item So the choice of initial weights---\emph{parameter
      initialisation}---matters.
  \item We'll see the same issue arise when we later look at another
    iterative algorithm: the EM algorithm.
  \item I've decided not to examine the various methods used for
    parameter initialisation in neural networks: it's enough that you
    know that it matters.
  \end{itemize}
  
\end{titledslide}
%%%%%%%%%%%%%%%%%%%%%%%%%%%%%%%%%%%%%%%%%%%%%%%%%%%%%%%%%%%%%%%%%%%%%%
\begin{titledslide}{Dealing with overfitting}

  \begin{itemize}
  \item Neural networks often have many parameters (ChatGPT has 175
    billion!) so overfitting is an issue.
  \item If we have a very large training dataset the problem is, at
    least, greatly reduced. (GPT-3.5 used 570GB of data, about 300
    billion words).
  \item But we are not always swimming in a vast sea of data!
  \item Let's look at some common ways of reducing overfitting in
    neural networks.
  \end{itemize}
  
\end{titledslide}
%%%%%%%%%%%%%%%%%%%%%%%%%%%%%%%%%%%%%%%%%%%%%%%%%%%%%%%%%%%%%%%%%%%%%%
\begin{titledslide}{Early stopping}

  \begin{itemize}
  \item Create a validation set.
  \item Stop training when error on the validation set starts to
    increase.
  \item (Of course, using a validation set to prevent overfitting is
    not unique to neural networks.)
  \end{itemize}
  
\end{titledslide}
%%%%%%%%%%%%%%%%%%%%%%%%%%%%%%%%%%%%%%%%%%%%%%%%%%%%%%%%%%%%%%%%%%%%%%
\begin{titledslide}{Weight decay}

  \begin{itemize}
  \item We can alter the loss function to penalise large weights.
  \item \emph{Weight decay}, also known as \emph{$\ell_2$
      regularisation}, adds a term $\lambda
    \mathbf{w}^{\top}\mathbf{w}$ to the loss function, where
    $\mathbf{w}$ is the weight vector and the user
    chooses the value of $\lambda > 0$.
  \item Murphy writes $\mathbf{w}^{\top}\mathbf{w}$ as
    $||\mathbf{w}||_{2}^{2}$ which makes it explicit that we are
    calculating (the square of) the 2-norm (aka Euclidean norm). See
    Murphy \S7.1.3 for more on norms if you're interested.
  \item In scikit-learn a parameter called \texttt{alpha} sets
    $\lambda$ to $\mathtt{alpha}/N$, where $N$ is the size of the
    training data. The default value of
    \texttt{alpha} is 0.0001, so we get a little \emph{$\ell_2$
      regularisation}.
  \item Since the loss we are trying to minimise is now not just about
    fitting the data, overfitting is reduced.
  \end{itemize}
  
\end{titledslide}
%%%%%%%%%%%%%%%%%%%%%%%%%%%%%%%%%%%%%%%%%%%%%%%%%%%%%%%%%%%%%%%%%%%%%%
\begin{titledslide}{Dropout}

  \begin{itemize}
  \item \emph{Dropout} is where, during training, we randomly turn of
    all outgoing connections from a unit with some given probability
    $p$.
  \item Do this on a per-example basis : ``For each presentation of each training case, a new thinned network is sampled and
trained.'' \cite{JMLR:v15:srivastava14a}
  \item Let's look at \cite[Fig~13.18]{pml1Book}.
  \item As Murphy puts it: `` \dots each unit must learn to perform well even
    if some of the other units are missing at random. This prevents
    the units from learning complex, but fragile, dependencies on each
    other.''
  \item Dropout can drastically reduce overfitting and is used a lot.
  \item Interesting comment from the inventors of dropout: ``\dots we would like to approach the performance of the
    Bayesian gold standard using considerably less computation.'' \cite{JMLR:v15:srivastava14a} 
  \end{itemize}
  
\end{titledslide}
%%%%%%%%%%%%%%%%%%%%%%%%%%%%%%%%%%%%%%%%%%%%%%%%%%%%%%%%%%%%%%%%%%%%%%
\begin{titledslide}{We're missing out a lot, as ever}

  \begin{itemize}
  \item There is of course much more to learn about neural networks.
  \item Most notably we have not looked into how the choice of
    \emph{architecture} (neural network structure) is made \dots
  \item \dots for example, convolutional NNs for image data.
  \item Also there has recently been a lot of attention on
    \emph{attention}: where the weights can depend on the input.
  \item We'll finish off our look at neural nets by going through some
    examples with PyTorch \dots
  \end{itemize}
  
\end{titledslide}
%%%%%%%%%%%%%%%%%%%%%%%%%%%%%%%%%%%%%%%%%%%%%%%%%%%%%%%%%%%%%%%%%%%%%%
\begin{titledslide}{Reading}

  \begin{itemize}
  \item Murphy The relevant sections of Chapter~13. 
  \end{itemize}
  
  
\end{titledslide}
%%%%%%%%%%%%%%%%%%%%%%%%%%%%%%%%%%%%%%%%%%%%%%%%%%%%%%%%%%%%%%%%%%%%%%
\begin{titledslide}{Problems and quizzes}

  \begin{itemize}
  \item No problems
  \item Quizzes:
    \begin{itemize}
    \item Week~2: Training Neural Networks
    \end{itemize}
  \end{itemize}
  
\end{titledslide}
%%%%%%%%%%%%%%%%%%%%%%%%%%%%%%%%%%%%%%%%%%%%%%%%%%%%%%%%%%%%%%%%%%%%%%
\bibliographystyle{alpha}
\bibliography{../ml.bib}


\end{document}
